\subsection{Glycosylated and Cysteine-rich Region }

\subsubsection{N-terminal region (aa108 - aa478)}

This domain defines DPP4 subfamily S9B, therefore is also referred as DPP IV N-terminal region. 

In terms of post-translational modifications, this region under heavy glycosylation, with up to 5 sites can be modified via N-Acetylglucosamine Asparagine (i.e. N-linked) at site 150, 219, 229, 281, 321.~\cite{Rasmussen2003,Thoma2003,Meng2010,Chen2009,Hiramatsu2003} N-glycosylation are believed important in protein folding and increasing structure stability.~\cite{Fan_1997} In addition, phosphorylation has been detected~\textit{in vivo} at site 256, 439, 440 in human sample.~\cite{Hornbeck2015, Mertins2014} There are also reported~\textit{in vivo} phosphorylation at site 211, 215, 219 in mouse leukemia T cell samples through mass spectrometry.~\cite{Hornbeck2015} 
Acetylation has also been reported in this region at site 137 (133 for mouse), 376 (372 for mouse), 434. ~\cite{Lundby2012,Weinert2013} Although there are studies that suggested glycosylation of DPP4 (CD26) is important in DPP4 co-activated T cells proliferation~\cite{Ikushima_2000} and phosphorylation is important for DPP4 signal conducting process~\cite{Ishii_2001}, the details of the exact mechanisms are not yet to be reviewed. 

\subsubsubsection{ADA binding site}
Leu294 and Val341 have been demonstrated via mutagenesis to be critical for ADA binding.~\cite{Abbott_1999} Leu294 and Val341 are located on two short $\alpha$-helices (291-294, 341-343 respectively), those two $\alpha$-helices are suspected to play important role in ADA binding. 